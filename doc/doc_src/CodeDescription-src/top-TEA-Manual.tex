\documentclass[letterpaper, 12pt]{article}

\usepackage{times}
\usepackage{margin}
\usepackage{natbib}
\usepackage{etoolbox}
\usepackage{astjnlabbrev-jh}
\usepackage{bibentry}
\usepackage{ifthen}
\usepackage{epsfig}

%\usepackage{graphicx}
\usepackage{enumitem}
\usepackage{amssymb, amsmath}
\usepackage{xcolor}
\usepackage{listings}
\usepackage{commath}
\usepackage{rotating}

\usepackage{dirtree}
\usepackage{changepage}

%\usepackage[T1]{fontenc}
%\usepackage[scaled]{beramono}
%\usepackage[utf8]{inputenc}
%\renewcommand*\familydefault{\ttdefault}

% \lstset{
% language=Python,
% showstringspaces=false,
% formfeed=\newpage,
% tabsize=4,
% commentstyle=\itshape,
% basicstyle=\ttfamily,
% morekeywords={models, lambda, forms}
% }
 
% \newcommand{\code}[2]{
% \hrulefill
% \subsection*{#1}
% \lstinputlisting{#2}
% \vspace{2em}
% }

% Default fixed font does not support bold face
\DeclareFixedFont{\ttb}{T1}{txtt}{bx}{n}{10} % for bold
\DeclareFixedFont{\ttm}{T1}{txtt}{m}{n}{10}  % for normal
% twelve-sized ttm:
\DeclareFixedFont{\tttb}{T1}{txtt}{bx}{n}{12}  % for bold
\DeclareFixedFont{\tttm}{T1}{txtt}{m} {n}{12}  % for normal

\DeclareFixedFont{\ttnm}{T1}{txtt}{m}{n}{9.8}  % for normal

% Custom colors
\usepackage{color}
\definecolor{deepblue}  {rgb}{0.0, 0.0, 0.5}
\definecolor{deepred}   {rgb}{0.8, 0.0, 0.0}
\definecolor{deepgreen} {rgb}{0.0, 0.5, 0.0}
\definecolor{commentc}  {rgb}{0.5, 0.5, 0.5}
\definecolor{DodgerBlue}{rgb}{0.1, 0.6, 1.0}

% Python style for highlighting
\newcommand\pythonstyle{\lstset{
language=Python,
basicstyle = \ttm,
morekeywords = {self, as, assert, with, yield}, % Add keywords here
keywordstyle  = \ttb\color{blue},      %
emph        = {MyClass, __init__},     % Custom highlighting
emphstyle   = \ttb\color{DodgerBlue},  % Custom  highlighting style
stringstyle = \color{deepred},         % Strings highlighting style
commentstyle=\color{commentc},         % Comment highlighting style
frame       = tb,                      % Any extra options here
showstringspaces = false
}}

% Python environment:
\lstnewenvironment{python}[1][]{\pythonstyle\lstset{#1}}{}
% Python for external files:
\newcommand\pythonexternal[2][]{{\pythonstyle\lstinputlisting[#1]{#2}}}
% Python for inline:
\newcommand\pythoninline[1]{{\pythonstyle\lstinline!#1!}}


% Python style for highlighting
\newcommand\plainstyle{\lstset{
language=Python,
basicstyle = \ttnm,
keywordstyle  = \ttnm,      %
emph        = {MyClass, __init__},     % Custom highlighting
emphstyle   = \ttnm\color{black},    % Custom  highlighting style
stringstyle = \color{black},         % Strings highlighting style
commentstyle=\color{black},         % Comment highlighting style
frame       = tb,                      % Any extra options here
showstringspaces = false
}}

% Plain environment:
\lstnewenvironment{plain}[1][]{\plainstyle\lstset{#1}}{}

\newcommand\plaininline[1]{{\plainstyle\lstinline!#1!}}


%\usepackage{fancyvrb}
%\usepackage{fixltx2e}
%\usepackage{pxfonts}
%\usepackage[tiny,compact]{titlesec}
%\usepackage{bera}
%\usepackage{alltt}
%\renewcommand{\ttdefault}{txtt}

% To use boldface verbatim:
%\lstset{basicstyle=\ttfamily,
%        escapeinside={||},
%        mathescape=true}

\lstset{
    language={[LaTeX]TeX},
    basicstyle=\tt\color{red},
    escapeinside={||},
}

\bibliographystyle{apj_hyperref}
\usepackage[%pdftex,      %%% hyper-references for pdflatex
bookmarks=true,           %%% generate bookmarks ...
bookmarksnumbered=true,   %%% ... with numbers
colorlinks=true,          % links are colored
citecolor=blue,           % green   % color of cite links
linkcolor=blue,           %cyan,         % color of hyperref links
menucolor=blue,           % color of Acrobat Reader menu buttons
urlcolor=blue,            % color of page of \url{...}
breaklinks=true,
linkbordercolor={0 0 1},  %%% blue frames around links
pdfborder={0 0 1},
frenchlinks=true]{hyperref}
%\usepackage{breakurl}

\newcommand{\eprint}[1]{\href{http://arxiv.org/abs/#1}{#1}}
\newcommand{\ISBN}[1]{\href{http://cosmologist.info/ISBN/#1}{ISBN: #1}}
\providecommand{\adsurl}[1]{\href{#1}{ADS}}

% hyper ref only the year in citations:
\makeatletter
% Patch case where name and year are separated by aysep:
\patchcmd{\NAT@citex}
  {\@citea\NAT@hyper@{%
     \NAT@nmfmt{\NAT@nm}%
     \hyper@natlinkbreak{\NAT@aysep\NAT@spacechar}{\@citeb\@extra@b@citeb}%
     \NAT@date}}
  {\@citea\NAT@nmfmt{\NAT@nm}%
   \NAT@aysep\NAT@spacechar\NAT@hyper@{\NAT@date}}{}{}
% Patch case where name and year are separated by opening bracket:
\patchcmd{\NAT@citex}
  {\@citea\NAT@hyper@{%
     \NAT@nmfmt{\NAT@nm}%
     \hyper@natlinkbreak{\NAT@spacechar\NAT@@open\if*#1*\else#1\NAT@spacechar\fi}%
       {\@citeb\@extra@b@citeb}%
     \NAT@date}}
  {\@citea\NAT@nmfmt{\NAT@nm}%
   \NAT@spacechar\NAT@@open\if*#1*\else#1\NAT@spacechar\fi\NAT@hyper@{\NAT@date}}
  {}{}
\makeatother


%\def\bibAnnoteFile#1{}
%\bibpunct[, ]{(}{)}{,}{a}{}{,}

% Packed reference list:
\setlength\bibsep{0pt}

% \textwidth=6.5in
% \textheight=9.5in
% \topmargin=-0.75in
% \oddsidemargin=0.0in
% \evensidemargin=0.0in

% \pagestyle{myheadings}
% \markright{MC\sp{3}}
% \pagenumbering{arabic}


% :::::::::::::::::::::::
\newcommand\degree{\degr}
\newcommand\degrees{\degree}
\newcommand\vs{\emph{vs.}}

% unslanted mu, for ``micro'' abbrev.
\DeclareSymbolFont{UPM}{U}{eur}{m}{n}
\DeclareMathSymbol{\umu}{0}{UPM}{"16}
\let\oldumu=\umu
\renewcommand\umu{\ifmmode\oldumu\else\math{\oldumu}\fi}
\newcommand\micro{\umu}
\newcommand\micron{\micro m}
\newcommand\microns{\micron}

\let\oldsim=\sim
\renewcommand\sim{\ifmmode\oldsim\else\math{\oldsim}\fi}
\let\oldpm=\pm
\renewcommand\pm{\ifmmode\oldpm\else\math{\oldpm}\fi}
\newcommand\by{\ifmmode\times\else\math{\times}\fi}
\newcommand\ttt[1]{10\sp{#1}}
\newcommand\tttt[1]{\by\ttt{#1}}
\newcommand\tablebox[1]{\begin{tabular}[t]{@{}l@{}}#1\end{tabular}}
\newbox{\wdbox}
\renewcommand\c{\setbox\wdbox=\hbox{,}\hspace{\wd\wdbox}}
\renewcommand\i{\setbox\wdbox=\hbox{i}\hspace{\wd\wdbox}}
\newcommand\n{\hspace{0.5em}}
\newcommand\marnote[1]{\marginpar{\raggedright\tiny\ttfamily\baselineskip=9pt #1}}
\newcommand\herenote[1]{{\bfseries #1}\typeout{======================> note on page \arabic{page} <====================}}
\newcommand\fillin{\herenote{fill in}}
\newcommand\fillref{\herenote{ref}}
\newcommand\findme[1]{\herenote{(FINDME: #1)}}

\newcount\timect
\newcount\hourct
\newcount\minct
\newcommand\now{\timect=\time \divide\timect by 60
         \hourct=\timect \multiply\hourct by 60
         \minct=\time \advance\minct by -\hourct
         \number\timect:\ifnum \minct < 10 0\fi\number\minct}

\newcommand\citeauthyear[1]{\citeauthor{#1} \citeyear{#1}}

\newcommand\mc{\multicolumn}
\newcommand\mctc{\multicolumn{2}{c}}


% {\tttm -h, --help} \\
% Print the list of arguments. \newline

% \newenvironment{myindentpar}[1]%
%   {\begin{list}{}%
%          {\setlength{\leftmargin}{3cm}}%
%      \item[]%
%   }
% {\end{list}}

\newenvironment{packed_enum}{
\begin{enumerate}[leftmargin=3cm]
   \setlength{\itemsep}{1pt}
   \setlength{\parskip}{5pt}
   \setlength{\parsep}{0pt}
}{\end{enumerate}}

\newcommand{\argument}[2]{{\noindent\tttm #1}%
\begin{adjustwidth}{2.5em}{0pt}%
#2 \vspace{0.3cm}%
\end{adjustwidth}%
}

%\newcommand{\ttmb}[1]{\tttm\color{#1}}
\newcommand{\routine}[2]{{\noindent\tttm\color{blue} #1:}%
\begin{adjustwidth}{2.0em}{0pt}%
#2 \vspace{0.15cm}%
\end{adjustwidth}%
}

% \newcommand{\routine}[2]{{\noindent\tttm\color{blue} #1}%
% \begin{adjustwidth}{0.0em}{0pt}%
%  #2 \vspace{0.15cm}%
% \end{adjustwidth}%
% }

% :::::::::::::::::: jhmacs2.tex :::::::::::::::::::::::::::::::::::::
\typeout{Joe Harrington's personal setup, Wed Jun 17 10:53:17 EDT 1998}
% Tue Mar 29 22:23:03 EST 1994

% :::::: pato.tex ::::::
% Joetex character unreservations.
% This file frees most of TeX's reserved characters, and provides
% several alternatives for their functions.


% utility
\catcode`@=11

% comments are first....
\newcommand\comment[1]{}

\newcommand\commenton{\catcode`\%=14}
\newcommand\commentoff{\catcode`\%=12}

% Not-a-comment:
\newcommand\nocomment[1]{#1}

\renewcommand\math[1]{$#1$}
\newcommand\mathshifton{\catcode`\$=3}
\newcommand\mathshiftoff{\catcode`\$=12}

\comment{alignment tab}
\let\atab=&
\newcommand\atabon{\catcode`\&=4}
\newcommand\ataboff{\catcode`\&=12}

\let\oldmsp=\sp
\let\oldmsb=\sb
\def\sp#1{\ifmmode
           \oldmsp{#1}%
         \else\strut\raise.85ex\hbox{\scriptsize #1}\fi}
\def\sb#1{\ifmmode
           \oldmsb{#1}%
         \else\strut\raise-.54ex\hbox{\scriptsize #1}\fi}
\newbox\@sp
\newbox\@sb
\def\sbp#1#2{\ifmmode%
           \oldmsb{#1}\oldmsp{#2}%
         \else
           \setbox\@sb=\hbox{\sb{#1}}%
           \setbox\@sp=\hbox{\sp{#2}}%
           \rlap{\copy\@sb}\copy\@sp
           \ifdim \wd\@sb >\wd\@sp
             \hskip -\wd\@sp \hskip \wd\@sb
           \fi
        \fi}
\def\msp#1{\ifmmode
           \oldmsp{#1}
         \else \math{\oldmsp{#1}}\fi}
\def\msb#1{\ifmmode
           \oldmsb{#1}
         \else \math{\oldmsb{#1}}\fi}
\def\supon{\catcode`\^=7}
\def\supoff{\catcode`\^=12}
\def\subon{\catcode`\_=8}
\def\suboff{\catcode`\_=12}
\def\supsubon{\supon \subon}
\def\supsuboff{\supoff \suboff}


\newcommand\actcharon{\catcode`\~=13}
\newcommand\actcharoff{\catcode`\~=12}

\newcommand\paramon{\catcode`\#=6}
\newcommand\paramoff{\catcode`\#=12}

\comment{And now to turn us totally on and off...}

\newcommand\reservedcharson{ \commenton  \mathshifton  \atabon  \supsubon
                             \actcharon  \paramon}

\newcommand\reservedcharsoff{\commentoff \mathshiftoff \ataboff \supsuboff
                             \actcharoff \paramoff}

\newcommand\nojoe[1]{\reservedcharson #1 \reservedcharsoff}

\catcode`@=12
\reservedcharsoff

\reservedcharson
\newcommand\jhauth[1]{{#1}}
\newcommand\jhstud[1]{{#1}}

\comment{Must have ONLY ONE of these... trust these macros, they work
\newcommand\jhauth[1]{{\bfseries #1}}
\newcommand\jhstud[1]{{\em #1}}
}

\reservedcharsoff
\reservedcharson


% ::::::::::::::::::::::::::::::::::::::::::::::::::::

\def\vs{{\em vs.}}
\def\p{\phantom{(0)}}

% Section levels:
\setcounter{secnumdepth}{5}
%  \section{}       % level 1
%  \subsection{}    % level 2
%  \subsubsection{} % level 3
%  \paragraph{}     % level 4 - equivalent to subsubsubsection
%  \subparagraph{}  % level 5

% To show in the table of content:
\setcounter{tocdepth}{5}

% Linebreak after \paragraph
\makeatletter
\renewcommand\paragraph{%
   \@startsection{paragraph}{4}{0mm}%
      {-\baselineskip}%
      {.5\baselineskip}%
      {\normalfont\normalsize\bfseries}}
\makeatother

% Linebreak after \paragraph
\makeatletter
\renewcommand\subparagraph{%
   \@startsection{subparagraph}{4}{0mm}%
      {-\baselineskip}%
      {.5\baselineskip}%
      {\normalfont\normalsize\bfseries}}
\makeatother

\actcharon
\renewcommand{\textfraction}{0.1}
\comment{\paramon\def\herenote#1{}\paramoff}
\renewcommand{\thepage}{\arabic{page}}
\reservedcharson

% :::::::::::: My Additions ::::::::::::::
\newcommand\Spitzer{{\em Spitzer}}
\newcommand\SST{{\em Spitzer Space Telescope}}
\newcommand\chisq{$\chi^2$}
\newcommand\itbf[1]{\textit{\textbf{#1}}}
\newcommand\bftt[1]{\texttt{\textbf{#1}}}
\newcommand\function[1]{\noindent\texttt{\begin{tabular}{@{}l@{}l}#1\end{tabular}}\newline}
\newcommand\bfv[1]{|\textbf{#1}|}
\newcommand\ttred[1]{\textcolor{red}{\ttfamily #1}}
\newcommand\ttblue[1]{\textcolor{blue}{\ttfamily #1}}
\newcommand\ttblack[1]{\textcolor{black}{\ttfamily #1}}
\newcommand\der{{\rm d}}
\newcommand\tno{$\sp{-1}$}
\newcommand\tnt{$\sp{-2}$}
\newcommand*\Eval[3]{\left.#1\right\rvert_{#2}^{#3}}
\newcommand\mcc{MC\sp{3}}
\newcommand\transit{{\tt Transit}}
\newcommand\pylineread{{\tttm Pylineread}}
%:::::::::::::::::::::::::::::::::::::::::
% Next six lines adjust spacing above/below captions and Sections etc
% Adjust as needed

\comment{
% \setlength{\abovecaptionskip}{0pt}
% \setlength{\belowcaptionskip}{0pt}
% \setlength{\textfloatsep}{8pt}
% \titlespacing{\section}{0pt}{5pt}{*0}
% \titlespacing{\subsection}{0pt}{5pt}{*0}
% \titlespacing{\subsubsection}{0pt}{5pt}{*0}
}

\reservedcharsoff
\actcharon
\mathshifton

\reservedcharson
